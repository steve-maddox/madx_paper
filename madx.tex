
\section{Source extraction}
\label{extract}

The ultimate aim for the source identification of the H-ATLAS data is
to use a multiband method to perform extraction across the five
wavebands simultaneously, thus utilising all the available data as
well as easily obtaining complete flux density information for each
detected galaxy, without having to match catalogues between
bands. However, the short timescale for the reduction of these SDP
observations, combined with the higher than expected PACS noise
levels, means that this was only possible for the three SPIRE
bands. As a result, the source extraction for the PACS and SPIRE maps
is discussed separately in this Section.

The full H-ATLAS SDP catalogue described here will be available at
\verb1http://www.h-atlas.org/1.

\subsection{The SPIRE catalogue}
\label{spire_cat}

Sources are identified in the SPIRE 250, 350 and 500\mic maps using
the Multi--band Algorithm for source eXtraction \citep[MADX][]{madx},
which is being developed for the H--ATLAS survey. The first step in
this method is to subtract a local background, estimated from the peak
of the histogram of pixel values in $30\times 30$ pixel blocks. This
corresponds to 2.5\arcmin $\times$ 2.5\arcmin for the 250\mic map, and
5\arcmin $\times$ 5\arcmin for the 350 and 500\mic maps. The
background (in mJy/beam) at each pixel was then estimated using a
bi-cubic interpolation between the coarse grid of backgrounds, and
subtracted from the data. Figure \ref{pre_backsub} illustrates the
reduction in backgound contamination (mainly arising from galactic
cirrus) obtained using this method.

The background subtracted maps were then filtered by the estimated
PSF, including an inverse variance weighting, where the noise for each
map pixel was estimated from the noise map (matched filtering).  The
maps from 350 and 500\mic bands are interpolated onto the 250\mic
pixels. Then all three maps are combined with weights set by the local
inverse variance, and also the prior expectation of the spectral
energy distribution (SED) of the galaxies. Each pixel in the resulting
combined map gives the log likelihood that there is a source with the
given SED at that pixel, so finding peaks in the combined map gives
the most likely positions of sources in the map. We tried a
flat-spectrum prior, where equal weight is given to each band, and
also 250\mic weighting, where only the 250\mic band was included. At
the depth of the filtered maps source confusion is a significant
problem, and the higher resolution of the 250\mic maps outweighed the
signal--to--noise gain from including the other bands (see Section
\ref{sim_creation} and Figure \ref{pos_err_pss_prior}). For our
current data we chose to use the 250$\mu $m only prior for all our
catalogues, which means that sources are identified at 250\mic
only. This may introduce a bias in the catalogue against red,
potentially high--redshift, sources that are bright at 500\mic, but
weak in the other bands. However, comparing catalogues made with both
priors showed that the number of missed sources is low: 2974
$>5\sigma$ 350\mic sources and 348 $>5\sigma$ 500\mic sources are
detected with the flat prior, compared with 2758 and 307 sources
detected using the 250\mic prior. We aim to revisit this issue in
future data--releases.

Local, $>2.5\sigma$, peaks are identified in the combined PSF filtered
map as potential sources, and sorted in order of decreasing
significance level. A Gaussian is fitted to each peak to provide an
estimate of the position at the sub--pixel level, as well as an
estimate of the peak value, which gives the best flux estimate for a
point source, though this can be influenced by the presence of a
neighbouring source, as illustrated in Figure \ref{cutout_plots}. The
scaled PSF is then subtracted from the map before going on to the next
source in the sequence.  The fluxes in other bands were estimated by
using a bi--cubic interpolation to the position given by the combined
map. Again a scaled PSF is subtracted from the map before estimating
the flux of the next source. This sorting and PSF subtraction reduces
the effect of confusion, but in future releases we plan to implement
multi-source fitting to blended sources.

To produce a catalogue of reliable sources, a source is only included
if it is detected at a significance of at least 5$\sigma$ in one of
the SPIRE bands. The total number of sources in the SPIRE catalogue is
6876.

In calculating the $\sigma$ for each source, we use the filtered noise
map and add the confusion noise to this in quadrature. The average
1$\sigma$ instrumental noise values are 4, 4 and 5.7 mJy/beam
respectively in the 250, 350 and 500\mic bands, determined from the
filtered maps. We estimated the confusion noise from the difference
between the variance of the maps and the expected variance due to
instrumental noise, and find that the 1$\sigma$ confusion noise is 5,
6 and 7 mJy/beam at 250, 350 and 500\mic; these values are in good
agreement with those found by \citet{nguyen} using data from the
Herschel Multi--tiered Extragalactic Survey (HerMES). The resulting
average 5$\sigma$ limits are therefore 33, 36 and 45 mJy/beam.
